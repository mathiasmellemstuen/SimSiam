\documentclass[12pt]{article}
\usepackage{lingmacros}
\usepackage{tree-dvips}
\usepackage{subfig}
\usepackage{graphicx}
\usepackage{multicol}
\usepackage{pythonhighlight}
\usepackage{amsmath}
\usepackage{geometry}
\usepackage{mathtools}
\usepackage{float}
\usepackage{hyperref}
\hypersetup{
    colorlinks=true,
    linkcolor=blue,
    filecolor=magenta,      
    urlcolor=cyan,
    pdftitle={Overleaf Example},
    pdfpagemode=FullScreen,
}
\DeclarePairedDelimiter{\abs}{\lvert}{\rvert}

\title{Implementation of the SimSiam algorithm}
\author{Mathias Mellemstuen}
\begin{document}

\bibliographystyle{plain} 

\maketitle

\section*{Introduction}
This report will explore the SimSiam \cite{chen2021exploring} algorithm. An implementation of SimSiam will be done and a SimSiam
network will be trained. The network will be trained with and without a stop-gradient operation and the results will be compared
with the results in \cite{chen2021exploring}.

\section*{Siamese Neural Networks}
Siamese neural networks is popular becuase of it's ability to make predictions based on a small training dataset. A Siamese
network contains two or more identical sub-networks, meaning the two instances of the network are sharing parameters and weights. When updating parameters,
the update should happen across both networks.

An undesired outcome of Siamese networks are collapsing of outputs. There are many different solutions to prevent collapsing,
like the ones used in other methods such as contrastive learning, clustering or BYOL. Simple Siamese Networks are using a simple method for
preventing collapsing which works well without utilizing any of the mentioned methods \cite{chen2021exploring}.

\section*{SimSiam}
SimSiam \cite{chen2021exploring} is a new method of Siamese Networks which prevents collapsing of the Siamese network in a simple way.
SimSiam is maximizing the similarity of two views in one image, without using one of the other
methods mentioned above. Collapsing solutions do exists in SiamSiam, and therefore a stop-gradient operation is
implemented in the algorithm to prevent collapsing.

The SimSiam algorithm was tested with and without the stop gradient operation. It was shown that
without the stop graident operation the network would collapse. This would happen just after a few epochs. It did not collapse
when running with the stop gradient operation. An empirical study was then done to examine if any other part like the predictor,
batch size and batch normalization could add to the contribution of preventing collapse. This emperical study concluded that
stop-gradient is the part of the algorithm that prevents collapsing. The other parts affected the accuracy, but did not show any
tendency to affect collapsing.

The SimSiam algorithm was compared against other state-of-the-art frameworks. This comparison showed that SimSiam has competitive results,
where the accuracy of SimSiam is the highest when doing under 100 epochs of pre-training. Other frameworks showed a higher accuracy when
training longer. When comparing SimSiam to SimCLR, SimSiam had better results in all cases.

The paper used the full 1000-class \textit{ImageNet} dataset without labels to train the network. The paper used a series of augmentations
like resize and cropping, horizontal flip, color jitter, grayscale and blurring. These augmentations were done with random parameters inside
a defined range for each augmentation. Two augmented views of the same image was then created and used further in the algorithm.

\section*{Implementation}
This implementation will only implement data loading, augmentation of the data and the SimSiam network.
Then the SimSiam network will be trained with and without the stop-gradient
part, to see if the results in figure 2 in \cite{chen2021exploring} will be replicated.

The dataset used in this implementation is the \textit{Tiny ImageNet (Stanford CS23N)} from \textit{ImageNet}. This dataset was chosen instead
of the full dataset from \textit{ImageNet} which was used in \cite{chen2021exploring}. This was done because there was a need to cut down
on computation time.

The augmentations performed are:
\begin{itemize}
    \item Resize and cropping
    \item Horizontal flip
    \item Color jitter
    \item Grayscale
    \item Blurring
\end{itemize}

The augmentations were done with the same parameters as in \cite{chen2021exploring}. These augmentations are done twice on
each image. This is important for getting two different views of the same image, which will be compared and used in the Siamese neural
network. The neural network is created with the same parameters as explained in \cite{chen2021exploring}.

\section*{Discussion}
The figure below shows the training loss when training the model for 50 epochs.

\begin{figure*}[h!]

    \centering
    \includegraphics*[width=0.9\textwidth]{figures/Figure_1.png}
    \caption{Visualizing training loss over 50 epochs of training, both with and without stop-gradient.}
    \label{results}
\end{figure*}

Figure \ref{results} shows that after just one or two epochs, the training loss (w/o stop-gradient) will converge to about -0.87. This is different from
the result in \cite{chen2021exploring} where the training loss would converge to -1.0 which is the minimum possible loss value. The loss with
stop-gradient has some
slight dissimilarities from the results in \cite{chen2021exploring}, but they both seem to be in the same area of around -0.7. One potential reason for the
dissimilarities in the results between figure 1 and figure 2 in \cite{chen2021exploring} is that the dataset in this implementation contains less data. The 
augmentations of the data is also random, meaning there will always be a slight difference in the results because the input data is
different.

It can further be seen that the stop gradient operation is helping with stopping the collapsing in this implementation. Without the stop gradient
operation, we can see that the training loss is converging to a loss of -0.87 after just a few epochs. This is indicating
that the stop-grad is an absolute necessary component of the algorithm to stop collapsing.

\section*{Conclusion}
After implementing the SimSiam algorithm and training the model for 50 epochs, the results showed some similarities with the results in
\cite{chen2021exploring} with some dissimilarities which should be expected. A difference from this implementation and the implementation in 
\cite{chen2021exploring} was probably expected, since less data was used and the augmentations are random. This implementation 
indicates that the stop-gradient operation is needed to stop collapsing.

\bibliography{citation}

\appendix
\section{Code}
The code for this project can be found \href{https://github.com/mathiasmellemstuen/SimSiam}{here}.
\end{document}